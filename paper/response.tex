\documentclass[12pt]{article}

\usepackage{amsmath}
\usepackage[margin=1in]{geometry}

\usepackage{quoting}

\usepackage[breaklinks=true]{hyperref}
\hypersetup{colorlinks,%
  citecolor=blue,%
  filecolor=blue,%
  linkcolor=blue,%
  urlcolor=blue}

% Set font of quotation to something different.
\usepackage{etoolbox}
\AtBeginEnvironment{quote}{\sf}

\usepackage{natbib}
\usepackage{aas_macros}
\bibliographystyle{aasjournal}

% \usepackage[utf8]{inputenc}
\usepackage[english]{babel}
 
\setlength{\parskip}{0.5em}

\begin{document}


\section*{Response to the referee's report}
\label{response-to-the-referees-report}

We thank the referee for his/her report and suggestions. The referee's
concerns and suggestions are largely based on Andrews et al. 2017, which
we do not agree with in many aspects. We have fully addressed the
referee's concerns and discussed aspects of this work that are directly
relevant when necessary.

We think the referee may be confusing three different levels of concerns
and is lumping them together as ``random alignments''. These are that
this pair may not physically associated

\begin{enumerate}
\def\labelenumi{\arabic{enumi}.}
\itemsep1pt\parskip0pt\parsep0pt
\item
  because it may be an optical pair, two stars that only happen to be
  close on sky
\item
  because the two stars may be only moving in the same direction on sky
\item
  because the two stars with such proximity in full 6D phase-space
  coordinates are still possible from a chance pair of unassociated
  field single stars.
\end{enumerate}

The first two concerns are already answered in the paper:

\begin{enumerate}
\def\labelenumi{\arabic{enumi}.}
\itemsep1pt\parskip0pt\parsep0pt
\item
  is addressed by that the two stars have nearly identical parallaxes
  meaning the two stars are physically close in position not just on
  sky. (e.g., section 3 in Poveda et al. 2009)
\item
  is addressed by that the two stars have nearly identical radial
  velocities with 0.2 km/s precision meaning that the two stars have
  very similar 3D velocity vector not just proper motion.
\end{enumerate}

The last concern is not the ``random alignments'' that Andrews et al.
2017 sample suffers from at projected separation above 0.1 pc nor
something addressed in that work.

Nonetheless the last concern is valid, and we have addressed this using the Gaia
Universe Mock Simulation, a mock end-of-mission Gaia catalog with realistic a
Milky Way model \citep{gums}, and still find this unlikely. We have added a new
section in Discussion, and acknowledged that such skepticism may still exist in
Summary.

We answer each of {\sf the referee's comments} in detail below.

\begin{quote}
Oh et al. calculate the likelihoods both that a pair is an associated
binary, and that it is a random alignment (an optical pair). They use a
model for the stellar components of the galaxy, which necessarily
involves assumptions. Thus, their list of binaries likely contains a
good number of false pairs, especially among those of projected
separations larger than about 10,000 AU (Andrews et al. 2017).
\end{quote}

While this is not a point of main discussion here, we note that this
summary of our work that the referee cites from Andrews et al. 2017 is
incorrect. We consciously did not overextend to say that we calculate
the likelihood that a pair is ``an associated binary'', and referred to
the candidates we found as candidate comoving pairs throughout the paper.
Nor do we calculate the likelihood that it is a `random alignment'.

\begin{quote}
My major concern is that the authors fail to make a convincing case for
both stars of the studied pair being physically associated. In fact, the
different chemical compositions can be viewed as a strong argument
against their physical association. Both stars do have nearly identical
proper motions and radial velocities, as well as similar parallaxes.
However, the statistical argument presented for their physical
association is not convincing for this particular pair.
\end{quote}

We would like to point out again that the two stars have \emph{maximum a
posteriori} separation of 0.6 pc and \emph{3D} velocity difference
consistent with zero. If the referee is concerned that the two stars
merely happen to lie closely on sky (`optical pair'), this is already
answered by the observation that the parallaxes of the two stars are
nearly identical. If the referee is concerned that the two stars only
happened to have nearly identical proper motions but not actually same
velocities, this concern is already answered by the observation that
their radial velocities measured to 0.2 km/s precision are also
identical.

\begin{quote}
The authors state that the velocity differences are consistent with
zero, but never mention the position differences.
\end{quote}

We point out that both the 3D and projected separation are explicitly
mentioned in the abstract and in the beginning of paragraph 3 in the
Data section where we discuss the similarity in their velocities.

\begin{quote}
The graph they show has a very poor spatial resolution, and cannot be
used to establish physical nearness.
\end{quote}

We think that in fact the contrary is true: the graph shows in all 6
dimensions how uncertain the differences in positions and velocities
are, and how the uncertainties are correlated. The graph reflects to the
fullest the uncertainties in parallaxes and proper motions as well as
their covariances.

\begin{quote}
In fact, the observed parallax difference alone implies a separation (in
the line of sight direction) of 140,275 AU, more than an order of
magnitude larger than the projected separation of the pair (12,198 AU).
Taking into account the parallax errors would allow even larger
line-of-sight separations. So, a random alignment cannot be excluded.
\end{quote}

The pair indeed has a larger radial separation than the projected
separation given current data, but we are not sure why that should
raise the suspicion of ``random alignments'' given that the 3D
velocities of the stars are very similar. We note that a) the 0.6 pc
separation we mention throughout the paper is the maximum a posteriori
estimation of separation, which corresponds to the peak of the
maginalized $p(\Delta x)$, $p(\Delta y)$, $p(\Delta z)$, and b) the
parallax uncertainties are not biases that goes in one direction to
increase the separation but may also go in the other direction to
decrease it although we do not think that our results are not contingent
on this.
What is presented is the best astrometric data available yet, and the span of
this uncertainty is fully displayed in Figure 1.

\begin{quote}
As shown by Andrews et al. (2017), equal radial velocities are a
necessary but not sufficient condition for true binarity.
\end{quote}

Here, the referee's concern is shifting from that the pair might be an
optical pair or that only their proper motions are consistent to that it
is still insufficient that the two stars are close in 3D separation and
3D velocity.

We do not agree that Andrews et al. showed equal radial velocities are a
necessary but not sufficient condition of true binarity (or more
appropriately, physical association). The basis of their claim, or what
might have convinced the referee of this statement in their work is
two-fold.

\begin{enumerate}
\def\labelenumi{\arabic{enumi}.}
\itemsep1pt\parskip0pt\parsep0pt
\item
  The ``random pairs'' that they select by calculating \emph{the
  probability of being a true binary} from the TGAS matched with the
  catalog shifted by certain amount in declination and both components
  of proper motions sometimes do have identical velocities although they
  ought to be all ``random alignments'' (Figure 9.)
\item
  The contamination rate that they estimate by matching RVs is higher
  than that by these ``random pairs'': ``Secondly, a consistent RV does
  not guarantee binarity; even systems with RVs consistent at the
  $3\sigma$ level may, in fact, be random alignments with similar, but
  slightly different RVs'' (par 5 Sec 5.2 in Andrews et al. 2017)"
\end{enumerate}

We do not find any of the two a convincing argument for the statement
as it may as well be a sign that there are shortcomings in
either their calculation of probability that a given pair forms a `true
binary'/`random alignment' or in the way they estimate the contamination
rates.

Because we cannot resolve the Keplerian orbits of such wide-separation comoving
pairs (as we argue below), if the alternative explanation to the physical
association of the two stars is that two random single stars could have ended up
in such configuration, this {\it requires} an assumption about the structure of
single-star-only Milky Way. This is not addressed in Andrews et al. 2017.


\begin{quote}
The authors need to provide an estimate of the contamination fraction
allowed by the parallax and radial velocity differences. It would be
illustrative to estimate the contribution of orbital motion (Gaia is
sensitive enough) and projection effects (the declination is
sufficiently large) to the observed differences, and to take these into
account in estimating the probability of the two stars being an optical
pair.
\end{quote}

A ``detection'' of orbital motion with a single-epoch astrometry of two
stars is ill-defined to begin with.
The magnitude of (3D) orbital velocity of a equal-mass binary of solar-mass
stars for binary semi-major axis of 0.01 pc (projected separation of this pair)
is 0.93 km/s,
which is an upper limit to the actual magnitude as the data
indicates that the 3D separation is likely larger as have been pointed
out by the referee as well.
Even if the proper motion difference is entirely from the orbital motion, some
(unknown) inclination of the orbital plane with respect to the line-of-sight
will only decrease the observable orbital velocity on sky.
Figure 1 is exactly what should urge the referee to re-think the claim of
Andrews et al. 2017 that the orbital motion can be detected for binaries with
orbital period as large as $10^6$ yrs (a=).
We note that the astrometric data in the two studies is identical.

Still we have considered the referee's concern that two unassociated
single field stars may have ended up in this close configuration in
phase-space coordinates. Because the orbital motion of such
wide-separation comoving pairs are not detectable, whether this
proximity in 6D phase-space coordinates, which is a naturally expected
outcome of coeval group of stars like wide binaries, is a very rare
configuration to be found among unassociated field stars or not
\emph{inevitably requires} an assumption about the Galactic distribution
function of single stars.

We used the Gaia Universe Mock Simulation (GUMS), a mock
end-of-mission Gaia catalog with realistic Milky Way model \citep{gums}.
Among 119259 solar-mass ($0.9~M_\odot< M < 1.1~M_\odot$) primary stars excluding
any companions to the primary, we find zero with differences in observed
quantities as small as this pair
(separation $<2$~pc,
$\Delta \mu_\alpha^* < 2$~mas\,yr$^{-1}$,
$\Delta \mu_\delta<2$~mas\,yr$^{-1}$ and
$\Delta v_r < 2$~km\,s$^{-1}$) still
allowing a larger difference than actually observed.
We find a single pair with velocity difference less than 2 km/s.
We conclude that this is an unlikely possibility, and 
have added a new section in Discussion (section 3.2).

\begin{quote}
Without a convincing argument for both stars being physically associated
the discussion about chemical abundance differences becomes largely
irrelevant.
The (evolutionary) ages of both stars are also similar, but have a large
uncertainty. Their similarity is not a convincing argument for true binarity.
In fact, the rotational velocities of both stars are very different,
which would argue for Kronos being younger than Krios. Although Li
abundances are a complicated matter, the surface Li abundance also
points to a younger age for Kronos. 
In summary, I cannot recommend
publication of the paper in its present form. I remain unconvinced about
the physical association of these stars. Perhaps the authors could
provide further arguments in favor of their true binarity (see eg.
Halbwachs 1986, Poveda et al. 2009), and thus make relevant the
interesting discussion about the abundance differences.
\end{quote}


We hope to have addressed all three levels of the referee's concern that the two
stars are unrelated.
We would also like to note that while having a comoving companion to anchor the
abundance differences to makes this pair even more interesting, the abundance
pattern of Kronos is quite unusual for the median expectation of differences
from unrelated single stars as we discuss in the last paragraph of section 2,
section 3.3 and Figure 6.

We discuss the possibility that Kronos is a very young star from many
perspectives in the last paragraph of section 3.1 and why we think this is not
the case.
The difference in $v\sin(i)$ could be due to a difference in inclinations or 
in initial angular momentum at formation. Or Kronos, which has a higher value,
could have spun up because of the angular momentum of the engulfed material.

Finally, we note that this particular pair is listed in the catalog of Andrews et al.
with ``true binary probability'' of 1.

% ask jmb: I think he said that he difference in values themselves are not
% significant aside from sin(i) factor.
% Also could have spun up.

\section*{Minor concerns}\label{minor-concerns}

\begin{quote}
\begin{enumerate}
\def\labelenumi{\arabic{enumi}.}
\itemsep1pt\parskip0pt\parsep0pt
\item
  The authors should be reminded that the IAU is the agency responsible
  for nomenclature of astronomical objects. They do refer to Kronos and
  Krios as ``nicknames'', but actually use them as names throughout the
  paper, as well as in the title. Using HD 240430/29 in the title makes
  the stars readily accessible in the astronomical databases, not so
  Kronos and Krios.
\end{enumerate}
\end{quote}
Thanks for reminding us but we will leave them as are. It is not our intention
to enforce the names in any way outside this publication. We refer to them as HD
identifiers in the abstract, introduction and summary as well as in keywords,
and we do not think the searchability of this work suffers.

\begin{quote}
\begin{enumerate}
\def\labelenumi{\arabic{enumi}.}
\setcounter{enumi}{1}
\itemsep1pt\parskip0pt\parsep0pt
\item
  Please define the entries in Table 1. Lower-case pi is often used for
  the radial galactocentric coordinate. Please include the separation
  (in arcsec) and the position angle.
\end{enumerate}
\end{quote}
We have defined all quantities besides the abundances in the caption of Table 1.

\begin{quote}
\begin{enumerate}
\def\labelenumi{\arabic{enumi}.}
\setcounter{enumi}{2}
\itemsep1pt\parskip0pt\parsep0pt
\item
  The pair is also listed in Halbwachs (1986)
\end{enumerate}
\end{quote}
Noted in sec 2 par 1.

\section*{Other changes}
\label{sec:other}

\begin{itemize}
\item Minor wording changes in the abstract.
\end{itemize}

%
\bibliography{ref}

\end{document}
