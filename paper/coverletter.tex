%% start of file `template.tex'.
%% Copyright 2006-2013 Xavier Danaux (xdanaux@gmail.com).
%
% This work may be distributed and/or modified under the
% conditions of the LaTeX Project Public License version 1.3c,
% available at http://www.latex-project.org/lppl/.
%Version for spanish users, by dgarhdez

\documentclass[12pt,letter,roman]{moderncv}


% moderncv themes
% style options are 'casual' (default), 'classic', 'oldstyle' and 'banking'
\moderncvstyle{classic}
% color options 'blue' (default), 'orange', 'green', 'red', 'purple', 'grey' and 'black'
\moderncvcolor{green}

% character encoding
\usepackage[utf8]{inputenc}

% adjust the page margins
\usepackage[scale=0.75]{geometry}

% personal data
\name{Semyeong}{Oh}
\address{4 Ivy Ln}{Princeton, NJ 08540}{United States}
\email{semyeong@astro.princeton.edu}


\begin{document}

% recipient data
\recipient{Dr. Leslie Sage}{
  968 National Press Building\\
  529 14th Street NW\\
  Washington DC 20045-1938\\
  United States}
\date{\today}
\opening{Dear Dr. Sage,}
\closing{Best regards,}
\makelettertitle

The enclosed paper presents clear evidence of massive accretion of rocky,
planetary material in the differential stellar chemical abundances of one star
in a comoving pair of sun-like stars. The other star in the pair also appears
to have accreted rocky material albeit in smaller amount. Figure~2 demonstrates
this key result: we show the abundance differences between the two stars
along with the change expected from accretion of bulk Earth material by a
sun-like star. We think that this result is the most convincing example of a
catastrophic end to planetary systems around sun-like stars, and thus will be
interesting to scientists in other disciplines as well as those in astronomy or
the exoplanets sub-field. We therefore submit the enclosed paper for
consideration for a Nature Letter.

The document has been formatted with the \LaTeX\ article class.
The summary is 286 words long, and the main text has $\approx 1467$ words, 29
references, two single-panel figures and one table, which should fit the page
limit of a Nature Letter.
We also include six extended data figures that will help interested readers
get the full context, as well as supplementary material discussing several
aspects of the main text in more detail.

\vspace{0.5cm}


\makeletterclosing

\end{document}
